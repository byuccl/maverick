\section{Related Work}
\label{sec:background}

The Maverick flow is a 3rd party CAD tool flow which targets commercial FPGA devices.
A number of other research projects have also produced tools in this same category, each of which addresses a different subset of the overall FPGA implementation flow.  
The tools most similar to the Maverick flow are described in this section.

Several tools have been introduced which allow for the manipulation of Xilinx designs in various ways.
Torc \cite{Steiner:2011}, RapidSmith \cite{Lavin:2011}, RapidSmith2 \cite{Haroldsen:2015}, and RapidWright \cite{Lavin:2018} all fit into this category---their principal use has been to enable users to perform CAD tasks for Xilinx FPGAs that are not readily possible using vendor tools \cite{Petelin:2016} \cite{Cannon:2018}.
Additionally, the VTR-to-Bitstream \cite{Hung:2015} project demonstrated  a 3rd party flow which implemented several steps of the flow for Xilinx devices.
All of these tools must return their designs to the Xilinx flow for the bitstream generation step.

In contrast, a number of other projects have been created that can generate bitstreams for commercial FPGAs.
The work in \cite{Steiner:2008} describes a proof-of-concept autonomous computing system running on the PowerPC of an embedded Virtex-II Pro device.
This system could place, route, and generate partial bitstreams for technology mapped (tech-mapped) designs. 
For bitstream generation, it used an unreleased version of Xilinx's JBits.

Another toolchain, the IceStorm flow \cite{IceStorm}, is a full FPGA CAD flow for the commercially available iCE40 family of FPGAs from Lattice Semiconductor.
It uses Yosys \cite{Wolf:2013} for synthesis, Arachne-pnr \cite{Arachne-pnr} for placement and routing, and the Project IceStorm tools for bitstream generation.
This flow has enabled Trenz Electronic's icoBoard \cite{icoBoard}, a Raspberry Pi accessory containing an iCE40 FPGA.
Using the icoBoard and a Raspberry Pi board, the IceStorm flow can compile designs to bitstreams on the Raspberry Pi's ARM CPU.
These bitstreams can then be programmed onto the iCE40 FPGA.

We believe that the Maverick work is interesting and novel because it combines the following characteristics.
First, it is a new PR flow which provides a new model for system development by enabling the independent creation of multiple RM designs once a static design has been created.
Furthermore, it is a lightweight CAD flow and can thus run on a single-chip embedded system in minimal memory (our demonstrations show it running in under 250 MB of RAM).
Also, being based on the RapidSmith/RapidSmith2 tool framework, it already supports Xilinx 7-Series devices and is readily extensible to future Xilinx devices such as UltraScale and UltraScale+.