\section{Conclusion and Future Work}
\label{sec:conclusion}

In this paper, we have presented the Maverick flow---a stand-alone CAD flow for RMs.
We demonstrated this by executing it on the PYNQ-Z1's ARM processor to compile a collection of Verilog designs to partial bitstreams.
These resulting bitstreams were all configured and verified in hardware on the PYNQ-Z1's FPGA fabric, demonstrating the feasibility of a single-chip system that can both compile HDL designs to bitstreams and then configure them onto its own programmable fabric.

Maverick uses a number of existing open source tools including Yosys, RapidSmith2, and Project X-Ray.  
We significantly modified some of these tools.
In particular, the existing RapidSmith2 import, packer, placer, and device model generation tools were all heavily modified to support PR, as was VDI.
Additionally, the Project X-Ray xc7Patch program was modified to create xc7PartialPatch, which functions with partial bitstreams.
New tools were also created as a part of this work, including a RapidSmith2-based router and a RapidSmith2 FASM file generator.
Other software was also created to interface the various pieces of Maverick to one another, creating a turnkey system.

We see several potential future opportunities for Maverick.
Firstly, we are preparing it for open source distribution so it can be used and extended by others in the community.
Maverick could also be enhanced to work with other Xilinx architectures beyond 7-Series, such as the Zynq UltraScale+.
There is nothing in the tools comprising Maverick which would prevent them from doing so.
In fact, several of these already have support for devices beyond 7-Series.
Another extension to Maverick would be to support multiple PR regions within the same device, a modest change to the existing system.
Maverick could also be updated to support additional primitives in 7-Series devices as Yosys and Project X-Ray expand their supported set.
Furthermore, we plan on exploring the use of Maverick in an educational setting to teach digital design, providing instructional materials, CAD tools, analysis and visualization tools, and a hardware platform in one PYNQ system.
